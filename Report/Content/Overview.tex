\section{Overview}

An important dependancy of the Ethereum PIR architecture relies in the ability to
privately retreive and process queries from a PIR database. Our implementation makes use
of
Spiral, which defines a family of single-server PIR protocols that relies on the
composition of two lattice-based homomorphic encryption schemes: the Regev encryption
scheme and the GentrySahai-Waters encryption scheme \cite{1}. Spiral proposes a range of
ciphertext translation techniques to convert between these two schemes and in doing so, is
able to achieve at least a 4.5x reduction in query size, 1.5x reduction in response size,
and a 2x increase in server throughput compared to previous systems\footnote{Previous
systems are defiend as SealPIR, FastPIR, MulPIR and OnionPIR. A complete and thorough
analysis between these systems may be found in the original paper \cite{1}}. The
\textit{"vanilla"} variant of Spiral is used throughout our implementation.

\subsection{Database}

The database of $N=2^{v_{1}+v_{2}}$ (where $v_{1}, v_{2} \in[2,11]$) records is arranged as a hypercube with dimensions
$2^{v_{1}} \times 2 \times 2 \times \cdots \times 2$. Processing the initial (large)
dimension requires scalar multiplication (given the database is known) and is implemented using matrix Regev
encodings. After processing the first dimension, the server has a $(2 \times 2 \times
\cdots \times 2)$-hypercube containing $2^{v_{2}}$ matrix-Regev encodings. The client's
index for each of the subsequent dimensions is encoded using GSW, so using $v_{2}$ rounds
of the Regev-GSW homomorphic multiplication, the server can "fold" the remaining elements
into a single matrix Regev encoding. We refer to Section 1.2 and Fig. 1 for a general
overview.

Database structure. Each database record $d_{i}$ is an element of $R_{p}^{n \times n}$, where $\left\|d_{i}\right\|_{\infty} \leq p / 2$. We represent a database $\mathcal{D}=\left\{d_{1}, \ldots, d_{N}\right\}$ of $N=2^{v_{1}+v_{2}}$ records as a $\left(v_{2}+1\right)$-dimensional hypercube with dimensions $2^{v_{1}} \times 2 \times 2 \times \cdots \times 2$. In the following description, we index elements of $\mathcal{D}$ using either the tuple $\left(i, j_{1}, \ldots, j_{v_{2}}\right)$ where $i \in\left[0,2^{v_{1}}-1\right]$ and $j_{1}, \ldots, j_{v_{2}} \in\{0,1\}$, or the tuple $(i, j)$ where $i \in\left[0,2^{v_{1}}-1\right]$ and $j \in\left[0,2^{v_{2}}-1\right]$.

\subsection{The Spiral Protocol}

The SPIRAL family of PIR protocols. The SPIRAL family of PIR protocol follows a similar high-level structure as previous lattice-based PIR protocols (Section 1.1). We describe the main steps here and also visually in Fig. 1:

- Query generation: The client's query consists of a single scalar Regev ciphertext that encodes the record index the client wants to retrieve. We structure the database of $N=2^{v_{1} \times v_{2}}$ records as a $2^{v_{1}} \times 2 \times \cdots \times 2$ hypercube. A record index can then be described by a tuple $\left(i, j_{1}, \ldots, j_{v_{2}}\right)$ where $i \in\left\{0, \ldots, 2^{v_{1}}-1\right\}$ and $j_{1}, \ldots, j_{v_{2}} \in\{0,1\}$. The query consists of an encoding of the vector $\left(i, j_{1}, \ldots, j_{v_{2}}\right)$, which we can pack into a single scalar Regev ciphertext using the Angel et al. [ACLS18] technique.

- Query expansion: Upon receiving the client's query, the server expands the query ciphertext as follows:

- Initial expansion: The server starts by applying the expansion technique from [ACLS18] to expand the query into a collection of (scalar) Regev ciphertexts that encode the queried index $\left(i, j_{1}, \ldots, j_{v_{2}}\right)$. This yields two collections of Regev ciphertexts, which we will denote by $C_{\mathrm{Reg}}$ and $C_{\mathrm{GSW}}$.

- First dimension expansion: Next, the server uses $C_{Reg}$ to expand the ciphertexts into a collection of $2^{v_{1}}$ matrix Regev ciphertexts that "indicate" index $i$ : namely, the $i^{th}$ ciphertext is an encryption of 1 while the remaining ciphertexts are encryptions of 0 . We can view this collection of ciphertexts as an encryption of the $i^{th}$ basis vector. This step relies on a scalar-to-matrix algorithm ScalToMat that takes a Regev ciphertext encrypting a bit $\mu \in\{0,1\}$ and outputs a matrix Regev ciphertext that encrypts the matrix $\mu \mathbf{I}_{n}$, where $\mathbf{I}_{n}$ is the $n$-by- $n$ identity matrix. We describe this construction in Section 3.1.

- GSW ciphertext expansion: The server then uses $C_{\mathrm{GSW}}$ to construct GSW
encryptions of the indices $j_{1}, \ldots, j_{v_{2}} \in\{0,1\}$. This step relies on a
Regev-to-GSW translation algorithm RegevToGSW that we describe in Section 3.2.

- Query processing: After expanding the query into matrix Regev encryptions of the first dimension and GSW encryptions of the subsequent dimension, the server follows the Gentry-Halevi blueprint [GH19] and homomorphically computes the response as follows:

- First dimension processing: First, it uses the matrix Regev encryptions of the $i^{th}$ basis vector to project the database onto the sub-database of records whose first index is $i$. This step only requires linear homomorphisms since the database records are available in the clear while the query is encrypted. At the end of this step, the server has matrix Regev encryptions of the projected database.

- Folding in subsequent dimensions: Next, the server uses the Regev-GSW external product to homomorphically multiply in the GSW ciphertexts encrypting the subsequent queries. Each GSW ciphertext selects for one of two possible dimensions. Since each multiplication involves a "fresh" GSW ciphertext derived from the original query, we can take advantage of the asymmetric noise growth property of Regev-GSW multiplication. The result is a single matrix Regev ciphertext encrypting the desired record.

- Response decoding: At the conclusion of the above protocol, the server replies with a single matrix Regev ciphertext encrypting the desired record.

We provide the full protocol description in Section 4 and a high-level illustration in Fig. 1.
